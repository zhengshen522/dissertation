\documentclass[
  11pt          % 
  ,letterpaper  %
  ,center       %
  ,noupper      %
  ]{uconnthesis2}
  
%\usepackage[
%	paper=letterpaper,
%	left=1.5in,
%	top=1in,
%	bottom=1in,
%	right=1in,
%	includehead,
%	includefoot
%]
%{geometry}
     
\voffset -15mm
\hoffset 2mm
\setlength{\textheight}{220mm}
\setlength{\textwidth}{145mm}
  
%\linespread{1.6}

  
\usepackage[table]{xcolor} %color keep it high
\usepackage{multicol} %multiple column
%\usepackage{times}
\usepackage[T1]{fontenc} %font
\usepackage{forest} %multidominance
\forestset{qtree edges/.style={for tree=
    {parent anchor=south, child anchor=north}}} %qtree edges
\usepackage{tikz-qtree} %tree
%\usepackage[utf8]{inputenc}
\usepackage{lmodern}
\usepackage{mathptmx}
\usepackage{epsfig}
\usepackage{natbib}
%\usepackage{fancyhdr}
%\pagestyle{fancy}
%\usepackage{geometry}
\usepackage{caption}
\usepackage{pifont}
\usepackage{tipa}


\usepackage{marvosym}
\usepackage{stmaryrd}
\usepackage{amssymb}
%\usepackage{textcomp}
\usepackage{pstricks}
\usepackage{colortab}
\usepackage{ulem} 
%\usepackage{yfonts}/Users/appleapple/Documents/Dropbox/OGPs/1. RNR/Conferences/PLC_39/PLC39_paper/gbpenn.sty
\usepackage{wasysym}
\usepackage{tree-dvips} %tree
\usepackage{qtree} %tree
\usepackage{enumerate} %lists
%\usepackage{gb4e}
\usepackage{linguex} %linguistics examples
%\usepackage{bbding}
%\usepackage{lscape}
%\usepackage{cite}
\usepackage{tikz}
\usetikzlibrary{arrows,calc}
\usepackage[colorlinks=true,linkcolor=blue,citecolor=blue]{hyperref}
%\usepackage{tocstyle}


\newcommand{\scell}[2][c]{%
  \begin{tabular}[#1]{@{}c@{}}#2\end{tabular}}
\newcommand{\cmark}{\ding{51}}
\newcommand{\xmark}{\ding{55}}
\renewcommand{\firstrefdash}{}    
\renewcommand{\theenumi}{\Alph{enumi}}
\setcounter{chapter}{3}
\newcommand\blueoline[1]{\colorlet{temp}{.}\color{blue}\overline{\color{temp}#1}\color{temp}}
\newcommand\redoline[1]{\colorlet{temp}{.}\color{red}\overline{\color{temp}#1}\color{temp}}
\newcommand\blueuline[1]{\colorlet{temp}{.}\color{blue}\underline{\color{temp}#1}\color{temp}}
\newcommand\reduline[1]{\colorlet{temp}{.}\color{red}\underline{\color{temp}#1}\color{temp}}
\setcounter{tocdepth}{2} %shows 3 levels excl. paragraph

%%%%%%%%%%%%%%%%%%%%%%

\usepackage[framemethod=TikZ]{mdframed}
\usepackage{xcolor}

% BOXES
%% set up the environment
\newmdenv[%
    backgroundcolor=red!8,
    linecolor=red,
    outerlinewidth=1pt,
    roundcorner=5mm,
    skipabove=\baselineskip,
    skipbelow=\baselineskip,
]{boxed}
%%%%%%%%%%%%%%%%%%%%%%

%delete above when compile

%updates 
%May. 5, 2017 - created
%May. 20, 2017 - line 389
%July. 21, 2017 - revise

\normalem

\title{Chapter 3: Experimental Studies on Nominal Right Node Raising}

\begin{document}

%\maketitle

%\chapter{Nominal Right Node Raising}

\noindent \large{\textbf{Chapter 3: Experimental Studies on Nominal Right Node Raising}} - Version 1 - Jul. 4, 2017\\

\noindent \large{\textbf{Zheng Shen}}

\tableofcontents

\section{Introduction}

In this chapter I present six experiments investigating the nominal right node raising construction. Last chapter lays out the detailed analysis of number markings on the pivot in nominal right node raising constructions. The empirical scope accounted for by the analysis can be divided into three types in \Next. 

\ex. 
\a. NRNR with Matching values: this tall and that short student are a couple.
\b. NRNR with mismatching values: These tall and that short student are a couple.
\c. NRNR with possessive pronouns: His and her student are a couple.

The core data the analysis is based on were collected via informal consultation with native speakers. This chapter uses experimental methods like 7 point Likert scale and forced choice tasks to substantiate the generalizations which the analysis in Chapter 2 is based on. The first three experiments look into the acceptability of singular and plural pivots under different sources. The fourth experiment specifically investigates the acceptability of the NRNR under possessive pronouns. The last two experiments investigate NRNR with mismatching values. 


Chapter 2 mentioned several investigations into nominals that receive multiple values (\citealt{Harizanov:2014, Harizanov:2015, Belyaev:2015, Shen:2016, Shen:2017, Shen:ta}). Among the previous research, no one has used experimental methods to probe the number marking of the pivot. So the studies presented in this chapter is the first of its kind. Note that a large part of the theoretical research on NRNR involve cross-linguistic variation. The experiments reported here only test English facts. I leave cross-linguistic experimental investigation for future research.

\section{Experiments on NRNR with Matching Values}

Experiment 1, 2, and 3 have a common goal: to test NRNR cases with matching values in English. Specifically, these three experiments check whether the generalization proposed in Chapter 2 repeated in \Next can be verified.

\ex.\label{ch3:gen}Generalization on NRNR: The singular pattern appears when the sources show morphological agreement with the pivot.

The generalization captures different number markings on the pivots in NRNR under different sources. The basic contrasts are shown in \Next - \NNext. In \Next, the sources \textit{this} and \textit{that} show number agreement, thus the pivot is marked as singular. In \NNext, the sources \textit{John's} and \textit{Mary's} do not show agreement, thus only the plural pivot is available. The following three studies aim check whether the contrasts hold in a controlled experiment.

\ex. 
\a. This and that student are a couple.
\b. *This and that students are a couple.

\ex.
\a. *John's and Mary's student are a couple.
\b. John's and Mary's students are a couple.

\subsection{Experiment 1 - Forced Choice Task 1}

\subsubsection{Materials, Participants and Procedure}

Experiment 1 is a forced choice task. In this task the subjects are presented with two minimally different sentences, and asked to choose the sentence that sounds more acceptable. In this particular experiment, the only factor is the number marking of the pivot noun and the two levels of the factor is singular and plural. Thus each target item includes two sentences with the NRNR construction, which only different in terms of the number marking of the pivot noun as in \Next. 

\ex. \a. This and that \textit{student} are a couple.
\b. This and that \textit{students} are a couple.

I test 11 different sources in English in total, listed in \Next. The labels of each source can be found in the parentheses. 

\ex.
\a. bare demonstratives (dem.non): This and that \textit{student(s)} are a couple.
\b. demonstrative+adj (dem.adj): This tall and that short \textit{student(s)} are a couple.
\c. numeral+adj (num.adj): One tall and one short \textit{student(s)} are a couple.
\d. indefinite article+adj (ind.adj): A tall and a short \textit{student(s)} are a couple.
\e. definite article+adj (def.adj): The tall and the short \textit{student(s)} are a couple.
\f. bare possessor DP (pos.non): John's and Mary's \textit{student(s)} are a couple.
\f. possessor DP+adj (pos.adj): John's tall and Mary's short \textit{student(s)} are a couple.
\f. bare 1st/2nd person possessive pronoun (pop12.non): my and your \textit{student(s)} are a couple.
\f. 1st/2nd person possessive pronoun+adj (pop12.adj): my tall and your short \textit{student(s)} are a couple.
\f. bare 3rd person possessive pronoun (pop33.non): her and his \textit{student(s)} are a couple.
\f. 3rd person possessive pronoun+adj (pop33.non): her tall and his short \textit{student(s)} are a couple.

4 lexically matched conditions are created for each of the 11 sources in order to avoid differences across conditions due to lexical content. Examples of the test items are listed below in \Next-\ref{exp1}. The conjoined DPs involving NRNR are the subject of the sentence. The predicates used ensure that the subjects refer to two individuals and thus each conjunct is singular. The order of the singular pivot sentences and the plural pivot sentences are counterbalanced. 

\ex. 
\a. This and that student are a couple.
\b. This and that students are a couple.

\ex. 
\a. This and that shoe were originally a pair.
\b. This and that shoes were originally a pair.

\ex.
\a. This and that kid are two of Emma's favorite teammates.
\b. This and that kids are two of Emma's favorite teammates.

\ex.\label{exp1} 
\a. This and that professor were the top two candidates.
\b. This and that professors were the top two candidates.

Each list contains 2 control items (\Next and \NNext) and one test item. Each control item also consists of two minimally different sentences. The control items are used to conceal the construction under investigation as well as to rule out people who are not paying attention to the content of the survey. The control items are the same in every list including the order of the sentences and the order of the items. They are taken from \citealt{Sprouse:2013}. The control items come before the test item. A total of 88 lists were compiled. The experiment was conducted on Amazon Mechanical Turk. Each list was done by 6 subjects. A total of 528 native speakers participated in the experiment. Each participant was paid 15 cents (5 cents per judgment).

\ex. control item 1: large difference
\a. *There might mice seem to be in the cupboard.
\b. There might seem to be mice in the cupboard.

\ex. control item 2: medium difference
\a. Into which room walked three men?
\b. *Into which room did walk three men?

\subsubsection{Results}

\paragraph{Results excluding 1 participant}~\\

Out of the 528 subjects who participated. Only 1 participant didn't pass the 1st control item. 61 participants didn't pass one or both control items. Since the two sentences in the 1st control item involves a large difference. The participant that didn't pass it was excluded in the analysis. A binomial test was run and the results are shown in Table \ref{tab:ch3exp1a}. The \textit{singular} column indicates the number of participants who chose the singular pivot over the plural pivot. The \textit{plural} column indicates the number of participants who chose the plural pivot over the singular pivot. Each source got 48 data points (except for dem.adj which has 47), indicated in the \textit{total} column. 

For sources like dem.non, dem.adj, num,adj, ind.adj, and pop12.adj, singular pivots are significantly preferred over plural pivots. Def.adj and pos.non show a significant preference for singular over plural. Pos.adj, pop12.non, pop33.non, pop33.adj did not show significant preferences. The yellow cells are the ones that participants significantly prefer.

\begin{table}[h] \small \centering
%\resizebox{0.5\textwidth}{!}
{
\begin{tabular}{ | c | c | c | c | c | c | c |} \hline 
sources		&	example 				& 	singular (count / \%)			&	plural (count / \%)	&	total	&	prediction 	&	p-value\\ \hline
dem.non		&	this and that N			&	\cellcolor{yellow}{42 / 91.7\%}	&	4 / 8.3\%			&	48	&	singular	&	\textbf{1.514e-09}	\\ \hline
dem.adj		&	this tall and that short N 	&	\cellcolor{yellow}{44 / 89.4\%}	&	5 / 10.6\%			&	47	&	singular	&	\textbf{2.458e-08}\\ \hline
num.adj		&	one tall and one short N	&	\cellcolor{yellow}{44 / 91.7\%}	&	4 / 8.3\%			&	48	&	singular	&	\textbf{1.514e-09}\\ \hline
ind.adj		&	a tall and a short N		&	\cellcolor{yellow}{45 / 93.8\%}	&	3 / 6.2\%			&	48	&	singular	&	\textbf{1.313e-10}\\ \hline
def.adj		&	the tall and the short	N	&	11 / 22.9\%				&	\cellcolor{yellow}{37 / 77.1\%}	&	48	&	\textcolor{red}{singular} &	\textbf{.0002}\\ \hline
pos.non		&	J's and M's N			&	4 / 8.3\%					&	\cellcolor{yellow}{44 / 91.7\%}	&	48	&	plural &	\textbf{1.514e-09}	\\ \hline
pos.adj		&	J's tall and M's short	N	&	18 / 37.5\%				&	30 / 62.5\%		&	48	&	\textcolor{red}{singular} &	.1114	\\ \hline
pop12.non	&	my and your N			&	25 / 52.1\%				&	23 / 47.9\%		&	48	&	n.s.		& .8854	\\ \hline
pop12.adj		&	my tall and your short N	&	36/ \cellcolor{yellow}75\%		&	12 / 25\%			&	48	&	singular	& \textbf{.0007}		\\ \hline
pop33.non	&	her and his N			& 	24 / 50\%					&	24 / 50\%			&	48	&	n.s.		& 1	\\ \hline
pop33.adj		&	her tall and his short	N	&	23 / 47.9\%				&	25 / 52.1\%		&	48	&	\textcolor{red}{singular} &	.8854	\\ \hline

%dem.non		&	***			&	1.514e-09		&	\cellcolor{yellow}44	&	4				&	48		\\ \hline
%dem.adj		&	*** 			&	2.458e-08		&	\cellcolor{yellow}42	&	5				&	47		\\ \hline
%num.adj		&	***			&	1.514e-09		&	\cellcolor{yellow}44	&	4				&	48		\\ \hline
%ind.adj		&	***			&	1.313e-10		&	\cellcolor{yellow}45	&	3				&	48		\\ \hline
%def.adj		&	***			&	0.0002222	&	11				&	\cellcolor{yellow}37	&	48		\\ \hline
%pos.non		&	***			&	1.514e-09		&	4				&	\cellcolor{yellow}44	&	48		\\ \hline
%pos.adj		&	n.s.			&	0.1114		&	18				&	30				&	48		\\ \hline
%pop12.non	&	n.s.			&	0.8854		&	25				&	23				&	48		\\ \hline
%pop12.adj		&	***			&	0.0007173	&	\cellcolor{yellow}36	&	12				&	48		\\ \hline
%pop33.non	&	n.s.			&	1			&	24				&	24				&	48		\\ \hline
%pop33.adj		&	n.s.			&	0.8854		&	23				&	25				&	48		\\ \hline
\end{tabular}}
\caption{Results for Experiment 1 with 1 subject excluded}
\label{tab:ch3exp1a}
\end{table} 


\paragraph{Results excluding 61 participants}~\\

Table \ref{tab:ch3exp1b} shows the analysis with a more strict criterion. It excludes 61 participants who didn't pass one or both of the control items. The more restricted criterion did not change the pattern of the results.

\begin{table}[htb!] \small \centering
%\resizebox{0.5\textwidth}{!}
{
\begin{tabular}{ | c | c | c | c | c | c | c |} \hline 
sources		&	example				& 	singular (count / \%)			&	plural (count / \%)	&	total		& prediction &	p-value \\ \hline
dem.non		&	this and that N			&	\cellcolor{yellow}40 / 91\%		&	4 / 9\%			&	44		& singular &	\textbf{1.71E-08}\\ \hline
dem.adj		&	this tall and that short N	&	\cellcolor{yellow}38 / 90.5\%	&	4 / 9.5\%			&	42		& singular &	\textbf{5.65E-08}\\ \hline
num.adj		&	one tall and one short N	&	\cellcolor{yellow}42 / 91.3\%	&	4 / 8.7\%			&	46		& singular &	\textbf{5.10E-09}\\ \hline
ind.adj		&	a tall and a short N		&	\cellcolor{yellow}36 / 92.3\%	&	3 / 7.7	\%			&	39	& singular &	\textbf{3.61E-08}\\ \hline
def.adj		&	the tall and the short	N	&	10 / 24.4\%			&	\cellcolor{yellow}31 / 75.6\%	& 41	& \textcolor{red}{singular}	&	\textbf{.0015}	\\ \hline
pos.non		&	John's and Mary's N		&	3 / 7.1\%				&	39 / \cellcolor{yellow}92.9\%	&	42	& plural 	&	\textbf{5.63E-09}\\ \hline
pos.adj		&	J's tall and M's short N	&	17 / 41.5\%				&	24 / 58.5\%		&	41	& \textcolor{red}{singular} &	.3489	\\ \hline
pop12.non	&	my and your N			&	23 / 56.1\%				&	18 / 43.9\%		&	41	& n.s.	&	.5327\\ \hline
pop12.adj		&	my tall and your short N	&	33 / \cellcolor{yellow}78.6\%	&	9 / 21.4\%			&	42	& singular &	\textbf{.0003}	\\ \hline
pop33.non	&	her and his N			&	20 / 50\%					&	20 / 50\%			&	40	& n.s.	&	1	\\ \hline
pop33.adj		&	her tall and his short	N	&	18 / 42.9\%				&	24 / 51.7\%		&	42	& \textcolor{red}{singular} &	.4408	\\ \hline
\end{tabular}}
\caption{Results for Experiment 1 with 61 subjects excluded} \label{tab:ch3exp1b}
\end{table} 

\subsubsection{Discussion}

The significant preference for singular in dem.non (\textit{this and that}), dem.adj (\textit{this tall and that short}), num.adj (\textit{one tall and one short}), and ind.adj (\textit{a tall and a short}) conditions are compatible with the generalization in \ref{ch3:gen} and are predicted by the analysis laid out in Chapter 2. As discussed in Chapter 2, the agreement between these sources and the pivot indicates a multi-dominance structure which generates singular pivots. The significant preference for plural pivots in pos.non (\textit{John's and Mary's}) is also predicted. The absence of agreement between possessor DPs and the possessee NP indicates a complex possessor structure which generates plural pivots. Pos.adj (\textit{John's tall and Mary's short}) shows no significant contrast. Def.adj (\textit{the tall and the short)} shows significant preference for plural pivots. These results are not predicted under the analysis proposed in Chapter 2. 

In terms of possessive pronouns, in Chapter 2 I argue that pop12 (\textit{my and your}) and pop33 (\textit{her and his}) with no adjectives are not acceptable regardless of the number marking on the pivot.  The forced choice task used in this experiment is sensitive to relative differences between two sentences. The lack of significant contrast between the singular and plural pivots here are expected if neither is acceptable. Further experiments are required to see whether these two cases are indeed unacceptable. Pop12.adj (\textit{my tall and your short}) shows significant preference for singular as is predicted, however, pop33.adj (\textit{her tall and his short}) which is also predicted to prefer singular did not show significant contrast between singular and plural.

Results from three sources are not predicted by the analysis; def.adj, pos.adj, and pop33.adj. These sources share the feature that the sources are not morphologically marked in terms of number. Two reasons are possible in accounting for the unpredicted results. First, participants on AMT might need more cue to ensure the intended interpretation of the DP (e.g. \textit{the tall and the short student(s)}). The intended interpretation for all the sentences in Experiment 1 is two individuals. Experiment 1 uses predicates like \textit{are a couple}, \textit{are the top two candidates}, etc,. to ensure this interpretation, however, the participants might have ignored the predicates and assume an interpretation where there are more than 2 individuals. Note if that happens, the string in \Next[a] would be accepted as opposed to \Next[b].

\ex. \a. The tall and the short students are ...
\b. *This and that students are ...

Second, the unpredicted results may involve a processing preference for locally grammatical (but globally ungrammatical) strings over strings that seem to be locally ungrammatical but are actually globally grammatical.\footnote{For discussion on locally grammatical strings, see \citealt{Tabor:2004}.} Take the def.adj condition as an example where the plural pivot is predicted to be bad but is judged better than the singular pivot. When the pivot is \textit{student} in \Next[a], the string \textit{student are} form a a local string that seems ungrammatical in isloation. In \Next[b] when the pivot is \textit{students}, the string \textit{students are} form a local string that seems grammatical. Since all the sentences in Experiment 1 involve the plural copula \textit{are}, the reasoning above might have contributed to the unexpected preference for plural over singular in these conditions. This is further supported by the fact that the unpredicted results (def.adj, pos.adj, pop33.adj) share the feature that the sources are not morphologically marked in terms of number. 

\ex. 
\a. The tall and \textcolor{blue}{the short} \reduline{\textcolor{blue}{student} are} ~a couple. (dispreferred) \footnote{The blue underline indicates locally grammatical strings of the pivot and the copula, the red underline indicates locally ungrammatical strings of the pivot and the copula. }
\b. *The tall and \textcolor{blue}{the short} \blueuline{\textcolor{blue}{students} are} ~a couple. (preferred)

The morphologically marked sources do not suffer from this processing preference. In \Next, the local strings involving the pivot and the copula are the same as in \Last. However, the local strings involving the second source and the pivot play a role. In \Next[a] the local string \textit{that student} is grammatical while the local string \textit{that students} is ungrammatical in \Next[b]. I postulate that the acceptability of the DP internal local string cancels out the unacceptability of the local string involving the head noun \textit{student} and the verb \textit{are}. \Next[a] is thus preferred over \Next[b] despite that the local string \textit{student are} in \Next[a] is ungrammatical. One possible reason for the local string \textit{that student} to cancel out the \text{student are}

\ex.
\a.  This and \textcolor{blue}{that} \reduline{\textcolor{blue}{student} are}~ a couple. (preferred)
\b. *This and \textcolor{red}{that} \blueuline{\textcolor{red}{students} are}~ a couple. (dispreferred)

To avoid these potential confounds, the materials used in the next experiment are modified accordingly.

%%%%%%%%%
\subsection{Experiment 2: Forced Choice Task 2}

\subsubsection{Materials, Participants, and Procedure}  

Experiment 2 is also a forced choice task. Like Experiment 1, the participants are asked to choose the more natural sounding one out of two minimally different sentences. To address the possible confounds that result in the unpredicted results in Experiment 1, two modifications have been made to Experiment 2.  First, to make sure that the participants judge the sentences based on the intended interpretation,  a context sentence setting the scenario up and a picture depicting the scenario are added to each pair of sentences. The participants can see directly that the intended interpretation of the DPs involving NRNR is two individuals. Second, the predicates of the test sentences have been replaced with predicates with verbs that are not morphologically number marked (e.g. \textit{came}), unlike \textit{are} in Experiment 1. This is to eliminate the possible processing effect outlined above.

Experiment 2 included 8 out of the 11 sources from Experiment 1. Dem.non (\textit{this and that}), dem.adj (\textit{this tall and that short}), num.adj (\textit{one tall and one short}), ind.adj (\textit{a tall and a short}) have shown clear preference for singular pivots in Experiment 1 as is predicted, so Experiment 2 only included ind.adj to make sure that the changes made do not effect the results for these sources. The rest of the sources are all included in Experiment 2: def.adj (\textit{the tall and the short}), pos.non (\textit{John's and Mary's}), pos.adj (\textit{John's tall and Mary's short}), pop12.non (\textit{my and your}), pop12.adj (\textit{my tall and your short}), pop33.non (\textit{her and his}), pop33.adj (\textit{her tall and his short}). 

Apart from the changes mentioned above and the smaller set of sources, the design of Experiment 2 is similar to Experiment 1. The only factor is the number marking on the pivot noun and the two levels are singular and plural. Two lexically matched conditions are created for each source in order to avoid differences across conditions due to lexical content as is in \Next. Within each item, the order of the singular pivot sentences and the plural pivot sentences are counterbalanced. 

\ex. 
\a. A tall and a short student(s) came from the U.S.
\b. A blue and a green book(s) fell on the table

Each list contains 2 control items (\Next and \NNext) and one test item \ref{fc2c}. The control items come before the test item. The control items are the same in every list including the order of the sentences and the order of the items. Each item consists of a context sentence, a minimal pair of sentences, along with one picture that depicting scenario, in that order. I would like to thank Yimei Xiang for allowing me to use her illustrations in the stimuli. 

\ex. Control item 1\\
The animals are having snacks.
\a. A bunny and two dogs is having ice cream.
\b. A bunny and two dogs are having ice cream.
\vspace{-0.5em}
\begin{figure}[htb!] 
\includegraphics[scale=0.13]{fc2a.jpeg} \centering
\caption{Exp 2: control item 1}
\label{fig:exp2a}
\end{figure}\vspace{-1em}

\ex. Control item 2\\
Bill, John, and Mary each have one child.
\a. Bill's child is wearing a red hat.
\b. John's child is wearing a red hat.
\vspace{-0.5em}
\begin{figure}[htb!]
\includegraphics[scale=0.13]{fc2b.jpeg} \centering
\caption{Exp 2: control item 2}
\label{fig:exp2b}
\end{figure}\vspace{-1em}

\ex.\label{fc2c} Sample test item\\
Mary and John advised two students each. The students have travelled here for an international conference.
\a. Mary's tall and John's short student came from the U.S.
\b. Mary's tall and John's short students came from the U.S.
\vspace{-0.5em}
\begin{figure}[htb!]
\includegraphics[scale=0.13]{fc2c.jpeg} \centering
\caption{Exp 2: sample test item}
\label{fig:exp2c}
\end{figure}
\vspace{-1em}

With 8 sources, 2 sentence orders, and 2 lexically matched conditions, a total of 32 lists were compiled. This experiment was conduct via google drive and the participants are recruited via the UConn Linguistics Subject Pool. 339 native speakers of English finished the experiment. 

\subsubsection{Results}

Out of the 399 native speakers that participated Experiment 2, four are excluded from the analysis since they failed the second control item (no one failed the first control item). 335 are included in the analysis. A binomial test was run and Table \ref{tab:ch3exp2a} summarizes the results of Experiment 2. The \textit{singular} column indicates the percentage of participants who chose the singular pivot over the plural pivot. The \textit{plural} column indicates the percentage of participants who chose the plural pivot over the singular pivot. Given the natural of the recruitment process, keeping the number of participants of each source exactly the same is hard. The numbers of participants for each source range from 35 to 46.



\begin{table}[h!] \small \centering
%\resizebox{0.5\textwidth}{!}
{
\begin{tabular}{ | c | c | c | c | c | c | c |} \hline 
sources		&	significance	&	p-value 		& 	singular				&	plural				&	Total	&	Prediction \\ \hline
ind.adj		&	***			&	1.336e-12		&	\cellcolor{yellow}97.8\%	&	2.2\%				&	46	&	singular \\ \hline
def.adj		&	\textcolor{red}{n.s}			&	0.7552		&	53.7\%				&	46.3\%	&	41	&	\textcolor{red}{singular} \\ \hline
pos.non		&	***			&	6.575e-05		&	20\%					&	\cellcolor{yellow}80\%	&	45	&	plural\\ \hline
pos.adj		&	significant		&	0.001914		&	\cellcolor{yellow}74.4\%	&	25.6\%				&	43	&	singular\\ \hline
pop12.non	&	n.s.			&	0.6358		&	55\%					&	45\%					&	40	&	n.s.\\ \hline
pop12.adj		&	***			&	0.0005083	&	\cellcolor{yellow}80\%	&	20\%					&	35	&	singular \\ \hline
pop33.non	&	marginal		&	0.02263		&	31.8\%	&	\cellcolor{yellow}68.2\% 	&	44	&	n.s.	\\ \hline
pop33.adj		&	***			&	0.0004309	&	\cellcolor{yellow}78\%	&	22\%					&	41	&	singular \\ \hline
%ind.adj		&	significant		&	1.336e-12		&	\cellcolor{yellow}45		&	1				&	46		\\ \hline
%def.adj		&	n.s			&	0.7552		&	22					&	19				&	41		\\ \hline
%pos.non		&	significant		&	6.575e-05		&	9					&	\cellcolor{yellow}36	&	45		\\ \hline
%pos.adj		&	significant		&	0.001914		&	\cellcolor{yellow}32		&	11				&	43		\\ \hline
%pop12.non	&	n.s.			&	0.6358		&	22					&	18				&	40		\\ \hline
%pop12.adj		&	significant		&	0.0005083	&	\cellcolor{yellow}28		&	7				&	35		\\ \hline
%pop33.non	&	marginal		&	0.02263		&	14					&	\cellcolor{yellow}30	&	44		\\ \hline
%pop33.adj		&	significant		&	0.0004309	&	\cellcolor{yellow}32		&	9				&	41		\\ \hline
\end{tabular}}
\caption{Results for Experiment 2} \label{tab:ch3exp2a}
\end{table} 

Ind.adj (\textit{a tall and a short}), pos.adj (\textit{John's tall and Mary's short}), pop12.adj (\textit{my tall and your short}), pop33.adj (\textit{her tall and his short}) reveal a significant preference for singular over plural. Pos.non shows a significant preference for plural over singular as Experiment 1. Pop33.non show marginal preference for plural over singular. Def.adj and pop12.non did not show significant contrast. 

\subsubsection{Discussion}

After the modifications, the results from Experiment 2 are more in line with the predictions made by the analysis in Chapter 2. The Pos.adj (\textit{John's tall and Mary's short}) condition which didn't show significant contrast between singular and plural now shows a significant preference for singular, as well as pop33.adj. 


This indicates that the proposed confounds in Experiment 1 (the overt plural marking on the verb and the context that is prone to be neglected) indeed interfere with the responses. Once these confounds are controlled, the analysis in Chapter 2 is further confirmed by the experimental results. Other conditions like ind.adj, pos.non, pop12.non, and pop33.non remain the same as Experiment 1 (if we interpret the marginal contrast of pop33.non as non significant), this indicates that the modification made in Experiment 2 did not compromise the `good' part of Experiment 1. The def.adj changes from significant preference for plural to non significant contrast. Although this is still not predicted in the theory, it is a move to the right direction. However it remains puzzling why def.adj is n.s when the analysis in Chapter 2 predicts a preference of singular over plural.

The next experiment focuses on the complex patterns observed on possessive pronouns in the NRNR construction.

%%%%%%%%%%%%%%%%%%%%%%%%%%
\subsection{Experiment 3: 7 point judgment task 1}

\subsubsection{Materials, Participants, and Procedure}

The forced choice task in Experiment 1 and 2 are sensitive to relative differences between two sentences. However, it does not distinguish situations where two sentences are both acceptable or both unacceptable. For example, difference between singular and plural pivots has been shown to be not significant under sources like pop12 (\textit{my and your}) and pop33 (\textit{her and his}) in Experiment 2. However, this does not speak to whether they are both acceptable or unacceptable. The analysis proposed in 2.3.4 in Chapter 2 predicts that both are unacceptable. To verify this prediction, Experiment 3 uses 7 point Likert scale task. In this task, the participants were given pairs of minimally different sentences and asked to rate the naturalness of each sentence by a rating between 1-7. 

The design of Experiment 3 is similar to that of Experiment 2. The only factor is the number marking on the pivots and the two levels are singular and plural. Two lexically matched conditions have been created and the order of the two sentences in the test items are counter-balanced. All 11 sources are included. 

Each list consists of two control items and one test item. Each item consists of one contextual sentence, one image depicting the scenario, and two minimally different sentences to judge in that order. The image is moved from after the minimal pair of sentences to between the context sentence and the minimal pair of sentences. This is to ensure that the participants pay attention to the scenario depicted in the image.\footnote{I would also like to thank Dorothy Ahn and Yimei Xiang for allowing me to use their illustrations in the stimuli.} The control items are the same across lists, shown in \Next and \NNext.

\ex. Control item 1\\
The animals are having snacks.\\
\vspace{-0.5em}
\begin{figure}[htb!] 
\includegraphics[scale=0.13]{7p1a.jpeg} \centering
\caption{Exp 3: control 1}
\label{fig:exp3a}
\end{figure}\vspace{-0.5em}
\a. A bunny and two dogs is having ice cream.
\b. A bunny and two dogs are having ice cream.

\ex. Control 2 \\
John is trying on his new red shorts.
\begin{figure}[htb!] 
\includegraphics[scale=0.13]{7p1b.jpeg} \centering
\caption{Exp 3: control 2}
\label{fig:exp3b}
\end{figure}\vspace{-0.5em}
\a. John is looking at himself in the mirror.	
\b. John is looking at him in the mirror.

\ex.\label{7p1c} Sample test item\\
Someone put several books on the table. The blue book belongs to Emily and the orange book belongs to Sarah.
\vspace{-0.5em}
\begin{figure}[htb!]
\includegraphics[scale=0.13]{7p1c.jpeg} \centering
\caption{Exp 3: test item}
\label{fig:exp3c}
\end{figure}
\vspace{-1em}
\a. Emily's blue and Sarah's orange book fell off the table.
\b. Emily's blue and Sarah's orange books fell off the table.

With 2 orders, 2 lexically matched conditions, and 11 sources, a total of 44 lists were compiled. The experiment was conducted on Amazon Mechanical Turk. Each list was done by 6 subjects and a total of 264 subjects finished the experiment. Each participant was paid 30 cents. 

\subsubsection{Results}

For the 7 point scale task, we show the results in two ways: we plot the responses and the mean judgments, and we show statistical results from a linear mixed effects modal. 

The results for each condition are plotted in Figure \ref{fig:ch3exp3.violin}. The white diamond indicates the mean judgment of that condition. The black dots indicates the individual judgments of that condition. The width of the colored bars (blue = singular, orange = plural) indicates the distribution of the judgments in each condition. %As we can see in dem.non, dem.adj, num.adj, ind.adj, pop12.adj, pop33.adj conditions, the mean judgments of the singular pivots are notably higher than the plural pivots. On the other hand, the mean judgment of plural pivots is significantly higher than that of singular pivots. Singular and plural pivots in def.adj, pos.adj, pop12.non, and pop33.non are marginally or less different.

\begin{figure}[h!] 
\includegraphics[scale=0.13]{plot_7p1_violin.png} \centering
\caption{Plot for Experiment 3}
\label{fig:ch3exp3.violin}
\end{figure}

\vspace{-0.5em}
A linear mixed effects model is created with the singular/plural pivot as the fixed factor. The results are summarized in Table \ref{tab:ch3exp3a}. Four pieces of information are provided: the mean rating (out of 7) for the sentences with singular pivots under the source (\textit{singular mean}), the mean rating (out of 7) with plural pivots under the source (\textit{plural mean}), the F value, and the p-value.

\begin{table}[htb!] \small \centering
\resizebox{0.8\textwidth}{!}
{
\begin{tabular}{ | c | c | c | c | c | c | c | c |} \hline 
sources		&	singular mean	&	plural mean	&	F value	&	p-value 		&	significance	&	Prediction \\ \hline
dem.non		&	4.9583		&	1.8333		&	58.409	&	< 0.0001		&	***			&	singular \\ \hline
dem.adj		&	4.7917		&	2.2084		&	27.526 	&	< 0.0001		&	***			&	singular \\ \hline
num.adj		&	5.8333		&	2.2083		&	71.611 	&	< 0.0001		&	***			&	singular\\ \hline
ind.adj		&	6.3333		&	2.125		&	148.21	&	< 0.0001		&	***			&	singular \\ \hline
\textcolor{red}{def.adj}		&	4.2500		&	5.2500		&	3.1908	&	0.0815	&	.	&	singular \\ \hline
pos.non		&	2.3333		&	6.4583		&	135.55	&	< 0.0001		&	***			&	plural \\ \hline
\textcolor{red}{pos.adj}		&	4.2083 		&	3.7916		&	0.5447	&	0.4642		&	n.s			&	singular \\ \hline
pop12.non	&	3.0417		&	2.8334 		&	0.1598 	&	0.6931		&	n.s			&	neither \\ \hline
pop12.adj		&	4.9167		&	2.9167		&	14.369	&	0.0004		&	***			&	singular \\ \hline
pop33.non	&	3.2917		&	3.2083		&	0.0293 	&	0.8649		&	n.s			&	neither \\ \hline
pop33.adj		&	4.8750		&	3.3333		&	9.2419	&	0.0058 		&	**			&	singular \\ \hline
\end{tabular}}
\caption{Results for Experiment 3} \label{tab:ch3exp3a}
\end{table} 

From both the plot and the statistics, we can see that dem.non, dem.adj, num.adj, ind.adj are compatible with Experiment 1 and 2: the singular pivot is acceptable, but not the plural pivot. Pos.non is also in line with the two previous experiments: the plural pivot is acceptable, but not the singular pivot. However, the pos.adj condition did not show significant contrasts between singular and plural pivots. Pop12.non and pop33.non did not show significant contrasts and in both cases the singular and the plural pivot are not accepted. In both pop12.adj and pop33.adj conditions, singular pivots are acceptable, plural pivots are not. In the def.adj condition, the plural pivot is marginally higher than the singular pivot. 

\subsubsection{Discussion}

From Experiment 3, several results from the previous experiments have been confirmed. For conditions like dem.non, dem.adj, num.adj, ind.adj, pop12,adj, and pop33.adj, the singular pivot is rated as acceptable. For pos.non, the plural pivot is acceptable. Experiment 3 reveals that both the singular and the plural pivot are unacceptable in pop12.non and pop33.non, just as \cite{Shen:ip2} predicted.

On the other hand, two conditions namely pos.adj and def.adj remain deviant from the predictions made by \cite{Shen:2016}. The pos.adj is predicted to prefer the singular pivot, however, Experiment 3 shows no significant contrast. This might result from the nature of the methodology: 7 point likert scale tasks have been shown to be less sensitive to subtle contrasts than forced choice tasks (\citealt{Sprouse:2013}). Note that the direction of preference for the pos.adj condition in Experiment 3 is predicted. Last but not the least, the def.adj condition shows a marginal preference for plural pivots. This has been a problem for Experiment 1-3. Note that informal judgement collection reveals an overwhelming preference for singular pivots. One possible cause for this difference between experiments carried online and in person survey is that the online group ignore the determiner \textit{the} in the second conjunct (due to inattention or attempting to finish the task fast). If that's what happens the intended sentences in \Next are coerced to sentences \NNext. Note that both \Next[a] and \NNext[b] are grammatical and felicitous in the scenario where there is one tall and one short student. Although the deviant responses can be accounted for this way, many questions remain: why participants of online surveys are more likely to coerce, what is the cause for this coerce, how to get rid of this process, and what this process reveals regarding the processing costs of multi-dominance.

\ex. 
\a. The tall and the short student came from the U.S.
\b. \#The tall and the short students came from the U.S.

\ex. 
\a. \#The tall and short student came from the U.S.
\b. The tall and short students came from the U.S.

\subsection{Summary of Experiments on NRNR with Matching Values}

Experiment 1, 2, and 3 use two methodology to probe the acceptance of different number markings on the pivot in NRNR. After dealing with confounds, the results largely confirm the analysis proposed in Chapter 2. A persistently deviant pattern has been observed in the condition where the definite determiner \textit{the} + adjective serve as the sources, as I note in last section, this may result from a strategy where some participants ignore the definite determiner in the second source, coercing the NRNR structure into a more frequently used structure.

\section{Experiment on possessive pronouns - Experiment 4}

%Why does it deserve a section
The three experiments reported above cover Nominal Right Node Raising construction with different sources. Experiment 4 focuses on NRNR with possessive pronouns as sources and look into whether the number marking on the pivot differs according to different combinations of possessive pronouns. \cite{Shen:ip2} reported that in nominal right node raising constructions, the possessive pronoun source in the first conjunct needs to be compatible with both overt and covert possessees, namely the 3rd person masculine singular possessive pronoun \textit{his}. This predicts that the singular pivot is accepted and preferred in the hisher.non condition, whereas neither the singular pivot nor the plural pivot is acceptable in conditions with other bare possessive pronouns in the first conjunct. All the conditions with adjectives, however, are predicted to show a preference of the singular pivot. The predictions are shown in \Next.

\ex.
\a. His and her student are a couple.
\b. *Her and his student are a couple.
\c. His tall and her short student are a couple.
\d. *His tall and her short students are a couple.
\e. Her tall and his short student are a couple.
\f. *Her tall and his short students are a couple.

\subsubsection{Materials}

Experiment 4 is a 7 point Likert scale. In this task, the participants were given pairs of minimally different sentences and asked to judge rate the naturalness of each sentence by a rating between 1-7. 

The design is similar to Experiment 3. The only factor is the number marking on the pivots and the two levels are singular and plural.  Two lexically matched conditions have been created for each condition and the order of the two sentence in the test items are counter-balanced. There are 12 sources in total in \Next.

%The hisher.non source is predicted to license singular target in NRNR while the herhis.non source do not. The hershis.non source probe the nominal possessive pronoun's inability to license plural/singular pivots. Pos.non and pos.adj are included to compare with possessive pronouns. 

\ex.
\a. hisher.non: His and her student(s) came from the U.S.
\b. hisher.adj: His tall and her short student(s) came from the U.S.
\c. herhis.non: Her and his student(s) came from the U.S.
\d. herhis.adj: Her tall and his short student(s) came from the U.S.
\e. hershis.non: Hers and his student(s) came from the U.S.
\f. hershis.adj: Hers tall and his short student(s) came from the U.S.
\f. pos.non: Mary's and John's student(s) came from the U.S.
\f. pos.adj: Mary's tall and John's short student(s) came from the U.S.
\f. yoursmy.non: Yours and my student(s) came from the U.S.
\f. yoursmy.adj: Yours tall and my short student(s) came from the U.S.
\f. myhis.adj: My and his student(s) came from the U.S.
\f. myhis.adj: My tall and his short student(s) came from the U.S.

Each list consists of two control items and one test item. Each item consists of one contextual sentence, one image depicting the scenario, and two minimally different sentences to judge in that order. The image is moved from after the minimal pair of sentences to between the context sentence and the minimal pair of sentences. This is to ensure that the participants pay attention to the scenario depicted in the image.2 The control items are the same across lists, shown in \Next and \NNext

\ex. Control item 1\\
The animals are having snacks.\\
\vspace{-0.5em}
\begin{figure}[htb!] 
\includegraphics[scale=0.13]{7p1a.jpeg} \centering
\caption{Exp 4: control item 1}
\label{fig:exp4a}
\end{figure}\vspace{-0.5em}
\a. A bunny and two dogs is having ice cream.
\b. A bunny and two dogs are having ice cream.

\ex. Control item 2 \\
John is trying on his new red shorts.
\vspace{-0.5em}
\begin{figure}[htb!] 
\includegraphics[scale=0.13]{7p1b.jpeg} \centering
\caption{Exp 4: control item 2}
\label{fig:exp4b}
\end{figure}\vspace{-0.5em}
\a. John is looking at himself in the mirror.	
\b. John is looking at him in the mirror.

\ex.\label{7p1c} Sample test item\\
Two professors and their students have travelled to France for an international conference. Professor Smith came from Canada and Professor Miller came from the U.K.
\vspace{-0.5em}
\begin{figure}[htb!]
\includegraphics[scale=0.13]{7p2c.jpeg} \centering
\caption{Exp 4: sample test item}
\label{fig:exp4c}
\end{figure}
\vspace{-1em}
\a. His and her student came from the U.S.
\b. His and her students came from the U.S.

\subsubsection{Participants and Procedure}

Given the 12 sources, 2 orders, and 2 lexically matched conditions, 48 lists are created. Each list is done by 6 participants (288 participants in total). This experiment is carried out on Amazon Mechanical Turk. Each participant is paid 30 cents. 

\subsubsection{Results}

For the 7 point scale task, I present the results in two ways:  the plots with the mean judgments, as well as statistical results from a linear mixed effects modal.

The results for each condition are plotted in Figure \ref{fig:ch3exp4noexl.violin}. The white diamond indicates the mean judgment of that condition. The black dots indicates the individual judgments of that condition. The width of the colored bars (blue = singular, orange = orange) indicates the distribution of the judgments in each condition.

\begin{figure}[h!] 
\includegraphics[scale=0.15]{plot_7p2_head_violin_noexl.png} \centering
\includegraphics[scale=0.15]{plot_7p2_tail_violin_noexl.png} \centering
\caption{Plot for Experiment 4 with no Exclusion}
\label{fig:ch3exp4noexl.violin}
\end{figure}

A linear mixed effects model is created with the singular/plural pivot as the fixed factor. The results are summarized in Table \ref{tab:ch3exp4a}. Four pieces of information is provide: the mean rating (out of 7) for the sentences with singular pivots under the source (singular mean), the mean rating (out of 7) with plural pivots under the source (plural mean), the F value, and the p-value.

\begin{table}[htb!] \small \centering
\resizebox{0.8\textwidth}{!}
{
\begin{tabular}{ | c | c | c | c | c | c |} \hline 
sources		&	singular mean	&	plural mean 	& 	F value		&	p-value		&	significance	\\ \hline
hisher.non		&	2.2			&	4.3			&	14.811 		&	0.0003647	&	***			\\ \hline
hisher.adj		&	4.4 			&	4.3			&	0.021957 		&	0.8828		&	n.s.			\\ \hline
herhis.non		&	2.5	 		&	4			&	8.2119 		&	0.006255		&	**			\\ \hline
herhis.adj		&	3.8 			&	3.2			&	1.0544 		&	0.3099		&	n.s.			\\ \hline
hershis.non	&	2.2 			&	3			&	2.5347  		&	0.125		&	n.s.			\\ \hline
hershis.adj	&	1.4 			&	1.5			&	2.0909 		&	0.1617		&	n.s.			\\ \hline
pos.non		&	2.2 			&	6			&	76.094 		&	2.6e-11		&	***			\\ \hline
pos.adj		&	4.1 			&	3.8			&	0.35105		&	0.5564		&	n.s.			\\ \hline
yoursmy.non	&	3.3 			&	3			&	0.2249 		&	0.6376		&	n.s.			\\ \hline
yoursmy.adj	&	2 			&	1.9 			&	0.12212 		&	0.7299		&	n.s.			\\ \hline
myhis.non		&	3.4 			&	3			&	0.59958 		&	0.4466		&	n.s.			\\ \hline
myhis.adj		&	5.1 			&	3.1			&	12.718 		&	0.0008588	&	***			\\ \hline
\end{tabular}}
\caption{Experiment 4 Results with no Exclusion } \label{tab:ch3exp4a}
\end{table} 

The above analysis includes every participant in Experiment 4 regardless how they did for control items. The following analysis excludes nine participants who had one or both of the control items wrong. Figure \ref{fig:ch3exp4b.violin} and Table \ref{tab:ch3exp4b} summarize the results. The exclusion did not affect the results in a significant way.

\begin{figure}[h!] 
\includegraphics[scale=0.15]{plot_7p2_head_violin.png} \centering
\includegraphics[scale=0.15]{plot_7p2_tail_violin.png} \centering
\caption{Plot for Experiment 4 with Exclusion}
\label{fig:ch3exp4b.violin}
\end{figure}

\begin{table}[htb!] \small \centering
\resizebox{0.8\textwidth}{!}
{
\begin{tabular}{ | c | c | c | c | c | c |} \hline 
sources		&	singular mean	&	plural mean 	& 	F value		&	p-value		&	significance	\\ \hline
hisher.non		&	2.2 			&	4.3			&	14.811 		&	0.0003647	&	***			\\ \hline
hisher.adj		&	4.4 			&	4.3			&	0.021957 		&	0.8828		&	n.s.			\\ \hline
herhis.non		&	2.5 			&	4			&	8.2119 		&	0.006255		&	**			\\ \hline
herhis.adj		&	3.8 			&	3.2			&	1.0544 		&	0.3099		&	n.s			\\ \hline
hershis.non	&	2 			&	2.9			&	2.8943  		&	0.103		&	n.s			\\ \hline
hershis.adj	&	1.4 			&	1.5			&	2.1 			&	0.1621		&	n.s			\\ \hline
pos.non		&	2.2 			&	6			&	76.094 		&	2.6e-11		&	***			\\ \hline
pos.adj		&	4.1 			&	3.8			&	0.35105		&	0.5564		&	n.s.			\\ \hline
yoursmy.non	&	3.4  			&	3.4			&	0      			&	1			&	n.s.			\\ \hline
yoursmy.adj	&	2			&	1.6			&	1.4177		&	0.2471		&	n.s.			\\ \hline
myhis.non		&	3.5 			&	3			&	0.59913 		&	0.4471		&	n.s.			\\ \hline
myhis.adj		&	5 			&	3			&	13.143 		&	0.0007452	&	***			\\ \hline
\end{tabular}}
\caption{Results for Experiment 4 with Exclusion} \label{tab:ch3exp4b}
\end{table} 

\subsubsection{Discussion}  

Several aspects of results of Experiment 4 are surprising.

First, the crucial prediction of \citealt{Shen:ip2} that the singular pivot is acceptable in hisher.non but not in herhis.non, is not borne out. Instead, hisher.non shows a significant preference for plural pivots, although the mean judgment (4.3/7) is lower than grammatical sentences (e.g. pos.non.pl: 5.6/7). On the other hand, althouh herhis.non is predicted to be unacceptable regardless of the number marking on the pivot, herhis.non sentences with a plural has a mean judgment of 4/7, which is borderline acceptable. Myhis.non which is similar to herhis.non is also predicted to be unacceptable for either number marking, this is borne out. The difference between the unpredicted behavior of herhis.non and the predicted behaviors of myhis.non is puzzling for the analysis laid out in section 2.3.4 in Chapter 2.

Second, some conditions with adjectives including herhis.adj, myhis.adj, pos.adj, and hisher.adj are predicted to show a singular preference. However only myhis.adj shows a significant preference. Both pos.adj and hisher.adj show no significant difference between singular and plural pivots. This could result from the fact that 7 point scale tasks are less sensitive to subtle differences, a point which was confirmed by pos.adj in Experiment 1 and 2.

Third, yoursmy.non, yoursmy.adj, hershis.non, hershis.non are included to test the claim in \citealt{Shen:ip2} that possessive pronouns that only can license covert possessees cannot be in the first conjunct in NRNR. This is born out. All the four conditions are not accepted. %However, the hershis.non and yoursmy.non do have higher mean than hershis.adj and yoursmy.adj. 

Last but not the least, two conditions in Experiment 4, herhis.non and herhis.adj, were also included in Experiment 3 using the same design and methodology (herhis.non/adj in Experiment 3 were labeled as \textit{pop33.non/adj}). However the results for these two conditions are different as is in Table \ref{tab:herhisexp3exp4}. In Experiment 3 herhis.non do not show a significant difference between the singular and plural pivots, where as in Experiment 4, there is a significant preference for plural pivots over singular ones. In Experiment 3, the herhis.adj show a significant preference for singular pivots, however no significant difference has been observed in Experiment 4. At this point, I do not have an explanation for this observation, especially given that the design, method, number of participants are the same between two experiments.

\begin{table}[htb!] \small \centering
\resizebox{0.8\textwidth}{!}
{
\begin{tabular}{ | c | c | c | c | c | c |} \hline 
sources			&	singular mean	&	plural mean 	& 	F value		&	p-value		&	significance	\\ \hline
herhis.non	 Exp 4	&	2.5 			&	4			&	8.2119 		&	0.006255		&	**			\\ \hline
herhis.non Exp 3	&	3.3			&	3.2			&	0.0293 		&	0.8649		&	n.s			\\ \hline
herhis.adj	Exp 4	&	3.8 			&	3.2			&	1.0544 		&	0.3099		&	n.s			\\ \hline
herhis.adj	Exp 3	&	4.9			&	3.3			&	9.2419		&	0.0058 		&	**			\\ \hline
\end{tabular}}
\caption{herhis.non and herhis.adj in Experiment 3 and 4} \label{tab:herhisexp3exp4}
\end{table} 

\textcolor{red}{I don't know what to do about this, maybe delete this entire experiment?}


%%%%%%%%%%%%%%%%%%%%%%%
%%%%%%%%%%%%%%%%%%%%%%%
%%%%%%%%%%%%%%%%%%%%%%%

\section{NRNR with mismatching values}

The experiments presented above aimed to test NRNR with two sources with the same number value, i.e. singular. The following two experiments aim to test NRNR with two sources with conflicting values, i.e. one singular and one plural. The analysis laid out in Chapter 2 predicts that in these cases the pivot noun will should the same number marking as the linearly closest source, i.e. closest conjunct agreement. 

\subsection{Experiment 5: forced choice task 3}

\subsubsection{Materials}

Experiment 5 is a forced choice task. Like Experiment 1 and 2, the participants were asked to choose the more natural sounding one out of two minimally different sentences. 

The only factor is the number marking on the pivot noun and the two levels are singular and plural. Two lexically matched conditions are created for each source in order to avoid differences across conditions due to lexical content. There are two variations of mismatches: one where the first conjunct is singular and the second is plural (SP) and one where the first conjunct is plural and the second is singular (PS). 8 sources are included: dem.non (\textit{these and that}), dem.adj (\textit{these tall and that short }), num.adj (\textit{two tall and one short}), def.adj (\textit{the tall and the short}), pos.non (\textit{John's and Mary's}), pos.adj (\textit{John's tall and Mary's short}), hisher.non (\textit{his and her}), hisher.adj (\textit{his tall and her short}). 

Each list contains 2 control items (\Next and \NNext) and one test item \ref{fc3c}. The control items come before the test item. The control items are the same in every list including the order of the sentences and the order of the items. Each item consists of a context sentence, a minimal pair of sentences, along with one picture that depicting scenario, in that order. I would like to thank Dorathy Ahn and Yimei Xiang for allowing me to use their illustrations in the stimuli. Within each item, the order of the singular pivot sentences and the plural pivot sentences are counterbalanced.

\ex. Control item 1\\
The animals are having snacks.\\
\vspace{-0.5em}
\begin{figure}[htb!] 
\includegraphics[scale=0.13]{7p1a.jpeg} \centering
\caption{Exp 5: control item 1}
\label{fig:exp5a}
\end{figure}\vspace{-0.5em}
\a. A bunny and two dogs is having ice cream.
\b. A bunny and two dogs are having ice cream.

\ex. Control item 2 \\
John is trying on his new red shorts.
\begin{figure}[htb!] 
\includegraphics[scale=0.13]{7p1b.jpeg} \centering
\caption{Exp 5: control item 2}
\label{fig:exp5b}
\end{figure}\vspace{-0.5em}
\a. John is looking at himself in the mirror.	
\b. John is looking at him in the mirror.

\ex.\label{fc3c} Sample test item\\
Three tall students and three short students have travelled to France for an international conference.
\vspace{-0.5em}
\begin{figure}[htb!]
\includegraphics[scale=0.13]{fc3c.jpeg} \centering
\caption{Exp 5: sample test item}
\label{fig:exp5c}
\end{figure}
\a. Two tall and one short student came from the U.S.
\b. Two tall and one short students came from the U.S.

\subsubsection{Procedure and Participants}

With 8 sources, 2 sentence orders, 2 lexically matched conditions and 2 variations of mismatch, there are 64 lists in total. 6 participants took each list. The experiment was conducted on Amazon Mechanical Turk. A total of 384 participants finished the survey. Each participant was paid 15 cents. 

\vspace{1cm}
\subsubsection{Results}

7 participants got 1 out of the 2 control items wrong and 1 participant got both control items wrong. All 384 participants were included in the analysis.

Table \ref{tab:ch3exp5a} summarizes the results from the binomial test on Experiment 5. The singular column indicates the percentage of participants who chose the singular pivot over the plural pivot. The plural column indicates the percentage of participants who chose the plural pivot over the singular pivot. The \textit{total} column notes the number of total participants for each condition. The results of Experiment 5 are divided into two types of mismatches: \{first conjunct plural-second conjunct singular (1P2S) \} and \{first conjunct singular-second conjunct plural (1S2P) \}.

For the 1S2P mismatch, a preference for plural pivots is observed for all the conditions. For the 1P2S mismatch on the other hand, this preference for plural pivots is observed for pos.non and hisher.non. Dem.non, dem.adj, num.adj show a preference for singular pivots. The remaining conditions (def.adj, pos.adj, hisher.adj) show no significant preference. 

\begin{table}[h!] \small \centering
%\resizebox{0.5\textwidth}{!}
{
\begin{tabular}{ | c | c | c | c | c | c |} \hline 
condition		&	result		&	p-value 		& 	singular				&	plural				&	total		\\ \hline
dem.non.ps	&	***			&	5.722e-06		&	\cellcolor{yellow}95.8\%	&	4.2\%			  	&	24		\\ \hline
dem.adj.ps	&	**			&	0.006611		&	\cellcolor{yellow}79.2\%	&	20.8\%				&	24		\\ \hline
def.adj.ps		&	.			&	0.02266		&	25\%					&	75\%					&	24		\\ \hline
num.adj.ps	&	***			&	0.001544		&	\cellcolor{yellow}83.3\%	&	26.7\%				&	24		\\ \hline
pos.non.ps	&	***			&	3.588e-05		&	8.3\%				&	\cellcolor{yellow}91.7\%	&	24		\\ \hline
pos.adj.ps		&	n.s			&	0.3075		&	37.5\%				&	62.5\%				&	24		\\ \hline
hisher.non.ps	&	***			&	0.0002772	&	12.5\%				&	\cellcolor{yellow}87.5\%	&	24		\\ \hline
hisher.adj.ps	&	n.s			&	0.3075		&	37.5\%				&	62.5\%				&	24		\\ \hline
~			&	~			&	~			&	~					&	~					&	~		\\ \hline
dem.non.sp	&	***			&	2.98e-06		&	4.2\%				&	\cellcolor{yellow}95.8\%	&	24		\\ \hline
dem.adj.sp	&	***			&	1.192e-07		&	0\%					&	\cellcolor{yellow}100\%	&	24		\\ \hline
def.adj.sp		&	***			&	0.0002772	&	12.5\%				&	\cellcolor{yellow}87.5\%	&	24		\\ \hline
num.adj.sp	&	***			&	1.192e-07		&	0					&	\cellcolor{yellow}100\%	&	24		\\ \hline
pos.non.sp	&	***			&	3.588e-05		&	8.3\%				&	\cellcolor{yellow}91.7\%	&	24		\\ \hline
pos.adj.sp		&	***			&	0.001544		&	26.7\%				&	\cellcolor{yellow}83.3\%	&	24		\\ \hline
hisher.non.sp	&	***			&	2.98e-06		&	4.2\%				&	\cellcolor{yellow}95.8\%	&	24		\\ \hline
hisher.adj.sp	&	***			&	0.0002772	&	12.5\%				&	\cellcolor{yellow}87.5\%	&	24		\\ \hline

%condition		&	result		&	p-value 		& 	singular				&	plural				&	total		\\ \hline
%dem.non.ps	&	***			&	5.722e-06		&	23					&	1					&	24		\\ \hline
%dem.adj.ps	&	**			&	0.006611		&	19					&	5					&	24		\\ \hline
%def.adj.ps	&	.			&	0.02266		&	6					&	18					&	24		\\ \hline
%num.adj.ps	&	***			&	0.001544		&	20					&	4					&	24		\\ \hline
%pos.non.ps	&	***			&	3.588e-05		&	2					&	22					&	24		\\ \hline
%pos.adj.ps	&	n.s			&	0.3075		&	9					&	15					&	24		\\ \hline
%hisher.non.ps	&	***			&	0.0002772	&	3					&	21					&	24		\\ \hline
%hisher.adj.ps	&	n.s			&	0.3075		&	9					&	15					&	24		\\ \hline
\end{tabular}}
\caption{Results for Experiment 5} \label{tab:ch3exp5a}
\end{table}

%\begin{table}[h!] \small \centering
%\resizebox{0.5\textwidth}{!}
%{
%\begin{tabular}{ | c | c | c | c | c | c |} \hline 
%condition		&	Significance 	&	p-value 		& 	singular	&	plural				&	total		\\ \hline
%dem.non.sp	&	***			&	2.98e-06		&	4.2\%	&	\cellcolor{yellow}95.8\%	&	24		\\ \hline
%dem.adj.sp	&	***			&	1.192e-07		&	0\%		&	\cellcolor{yellow}100\%	&	24		\\ \hline
%def.adj.sp		&	***			&	0.0002772	&	12.5\%	&	\cellcolor{yellow}87.5\%	&	24		\\ \hline
%num.adj.sp	&	***			&	1.192e-07		&	0		&	\cellcolor{yellow}100\%	&	24		\\ \hline
%pos.non.sp	&	***			&	3.588e-05		&	8.3\%	&	\cellcolor{yellow}91.7\%	&	24		\\ \hline
%pos.adj.sp		&	***			&	0.001544		&	26.7\%	&	\cellcolor{yellow}83.3\%	&	24		\\ \hline
%hisher.non.sp	&	***			&	2.98e-06		&	4.2\%	&	\cellcolor{yellow}95.8\%	&	24		\\ \hline
%hisher.adj.sp	&	***			&	0.0002772	&	12.5\%	&	\cellcolor{yellow}87.5\%	&	24		\\ \hline

%dem.non.sp	&	***			&	2.98e-06		&	1		&	23					&	24		\\ \hline
%dem.adj.sp	&	***			&	1.192e-07		&	0		&	24					&	24		\\ \hline
%def.adj.sp	&	***			&	0.0002772	&	3		&	21					&	24		\\ \hline
%num.adj.sp	&	***			&	1.192e-07		&	0		&	24					&	24		\\ \hline
%pos.non.sp	&	***			&	3.588e-05		&	2		&	22					&	24		\\ \hline
%pos.adj.sp	&	***			&	0.001544		&	4		&	20					&	24		\\ \hline
%hisher.non.sp	&	***			&	2.98e-06		&	1		&	23					&	24		\\ \hline
%hisher.adj.sp	&	***			&	0.0002772	&	3		&	21					&	24		\\ \hline
%\end{tabular}}
%\caption{Singular Plural Results for Experiment 5} \label{tab:8}
%\end{table}

\subsubsection{Discussion}

According to the analysis in Chapter 2, regardless of the mismatch types, pos.non is predicted to allow plural pivot. This prediction is borne out. 

The rest of the conditions involve genuine NRNR construction, which should show closest conjunct agreement, i.e. in 1S2P, the plural pivot is preferred, in 1P2S, the singular pivot is preferred. This prediction is confirmed for 1S2P cases. All the sources in 1S2P cases show a significant preference for plural pivots. However this is not surprising, since the reference of the conjoined DPs is plural and the morphological marking on the second source is also plural. It could be the case that mismatch in NRNR is resolved not by closest conjunct agreement but by plural marking. 

The crucial case is 1P2S where the singular pivot is predicted to be acceptable despite that the reference of the conjoined DPs is plural. This prediction is borne out for some of the sources dem.non, dem.adj, and num.adj. On the other hand, pos.adj, hisher.adj, and def.adj show no significant preference. Some conditions even show marginal preference for the plural pivot. Note that in 1P2S mismatch, the sources that show CCA are morphologically marked and the ones that do not show significant difference are unmarked. It is possible that CCA requires or prefers to be triggered by morphological marking.\footnote{It is also possible that the observed CCA is a result from agreement attraction.}

Finally, hisher.non.sp shows a significant preference for plural pivots, similar to pos.non.sp. This is not predicted under the analysis proposed by \cite{Shen:ip2}. However note that Experiment 4 shows that hisher.non with matching values already shows a preference for plural pivots. If we assume that at least part of the speakers treat possessive pronouns like \textit{his} on the par with possessor DPs like \textit{John's}, this parallelism is accounted for. Data from both matching and mismatching values converge to support this possibility. 

%%%%%%%%%%%%%%%%%%%%%%%%%%
%%%%%%%%%%%%%%%%%%%%%%%%%%
\subsection{Experiment 6: 7p Likert scale task}

\subsubsection{Materials}

Experiment 6 is a 7 point Likert scale task. The participants are asked to rate the naturalness of sentences with a 7 point scale, 1 being the least natural and 7 being the most natural. The materials are identical to those in Experiment 5 except for the methodology.

\subsubsection{Procedure and Participants}

With 8 sources, 2 sentence orders, 2 lexically matched conditions and 2 variations of mismatch (1P2S, 1S2P), 64 lists are created in total. 6 participants took each list. The experiment was conducted on Amazon Mechanical Turk. A total of 384 participants finished the survey. Each participant was paid 30 cents. 

\subsubsection{Results}

16 participants had one or both of the control items wrong. Excluding these 16 participants does not make a significant difference to the results. The following analysis includes all 384 participants.

For the 7 point scale task, we show the results in two ways: we plot the responses and the mean judgments, and we show statistical results from a linear mixed effects modal. The results for each condition are plotted in Figure \ref{fig:ch3exp6b.violin}. The white diamond indicates the mean judgment of that condition. The black dots indicates the individual judgments of that condition. The width of the colored bars (blue = singular, orange = orange) indicates the distribution of the judgments in each condition.

A linear mixed effects model is created with the singular/plural pivot as the fixed factor. The results are summarized in Table \ref{tab:ch3exp6a}. The results are divided into two types of mismatches: {first conjunct plural-second conjunct singular (1P2S)} and {first conjunct singular-second conjunct plural (1S2P)}. Four pieces of information is provide: the mean rating (out of 7) for the sentences with singular pivots under the source (singular mean), the mean rating (out of 7) with plural pivots under the source (plural mean), the F value, and the p-value.

\begin{figure}[h!] 
\includegraphics[scale=0.15]{plot_7p3_head_violin_judgment.png} \centering
\includegraphics[scale=0.15]{plot_7p3_tail_violin_judgment.png} \centering
\caption{Plot for with no exclusion}
\label{fig:ch3exp6b.violin}
\end{figure}

\vspace{-1em}
\begin{table}[htb!] \small \centering
\resizebox{0.8\textwidth}{!}
{
\begin{tabular}{ | c | c | c | c | c | c |} \hline 
sources			&	singular mean	&	plural mean 	& 	F value		&	p-value		&	Significance	\\ \hline
dem.non.sp		&	1.3			&	2.8		&	13.963		&	0.001079		&	**			\\ \hline
dem.adj.sp		&	1.7			&	4.7		&	33.586		&	5.862e-07		&	***			\\ \hline
num.adj.sp		&	1.7			&	6.4		&	186.14		&	2.2e-16		&	***			\\ \hline
def.adj.sp			&	3.6			&	5.8		&	15.005		&	0.0003377	&	***			\\ \hline
pos.non.sp		&	2.8			&	5.6		&	43.991		&	3.247e-08		&	***			\\ \hline
pos.adj.sp			&	3			&	5.8		&	32.522		&	8.057e-07		&	***			\\ \hline
hisher.non.sp		&	2.1 			&	5.2		&	71.077		&	1.724e-08		&	***			\\ \hline
hisher.adj.sp		&	3			&	5.4		&	20.007		&	5.028e-05		&	***			\\ \hline
~				&	~			&	~		&	~			&	~			&	~			\\ \hline
dem.non.ps		&	3.8			&	1.5		&	20.323		&	4.484e-05		&	***			\\ \hline
dem.adj.ps		&	4.3 			&	2.3		&	10.204		&	0.002531		&	**			\\ \hline
num.adj.ps		&	5.2			&	3		&	15.713		&	0.000255		&	***			\\ \hline
def.adj.ps			&	3.9			&	5.6		&	10.819		&	0.001931		&	**			\\ \hline
pos.non.ps		&	1.9			&	5.8		&	93.525		&	1.432e-09		&	***			\\ \hline
pos.adj.ps			&	4.3			&	4.3		&	0.019377		&	0.8899		&	n.s.			\\ \hline
hisher.non.ps		&	2.3			&	5.2		&	32.522		&	8.057e-07		&	***			\\ \hline
hisher.adj.ps		&	4.1			&	4.3		&	0.17322		&	0.6792		&	n.s.			\\ \hline

%sources			&	zscore F value	&	zscore p-value 	& 	raw F value	&	raw p-value	\\ \hline
%dem.non.sp		&	20.726		&	3.878e-05		&	13.963		&	0.001079		\\ \hline
%dem.adj.sp		&	45.222 		&	2.36e-08		&	33.586		&	5.862e-07		\\ \hline
%num.adj.sp		&	260.76 		&	2.2e-16		&	186.14		&	2.2e-16		\\ \hline
%def.adj.sp		&	26.309		&	5.69e-06		&	15.005		&	0.0003377	\\ \hline
%pos.non.sp		&	66.09		&	1.906e-10		&	43.991		&	3.247e-08		\\ \hline
%pos.adj.sp		&	55.335		&	2.01e-09		&	32.522		&	8.057e-07		\\ \hline
%hisher.non.sp		&	97.512 		&	9.69e-10		&	71.077		&	1.724e-08		\\ \hline
%hisher.adj.sp		&	24.032		&	1.22e-05		&	20.007		&	5.028e-05		\\ \hline
\end{tabular}}
\caption{Results for Experiment 6 with no exclusion} \label{tab:ch3exp6a}
\end{table} 
%\begin{table}[htb!] \small \centering
%\resizebox{0.8\textwidth}{!}
%{
%\begin{tabular}{ | c | c | c | c | c | c |} \hline 
%sources		&	singular mean	&	plural mean 	& 	F value		&	p-value	&	Significance	\\ \hline
%dem.non.ps	&	3.7500		&	1.5417		&	20.323		&	4.484e-05	&	***			\\ \hline
%dem.adj.ps	&	4.25 			&	2.2917		&	10.204		&	0.002531	&	**			\\ \hline
%num.adj.ps	&	5.2083		&	3.0416		&	15.713		&	0.000255	&	***			\\ \hline
%def.adj.ps		&	3.875		&	5.625		&	10.819		&	0.001931	&	**			\\ \hline
%pos.non.ps	&	1.875		&	5.75			&	93.525		&	1.432e-09	&	***			\\ \hline
%pos.adj.ps		&	4.25			&	4.3333		&	0.019377		&	0.8899	&	n.s.			\\ \hline
%hisher.non.ps	&	2.2917 		&	5.1667		&	32.522		&	8.057e-07	&	***			\\ \hline
%hisher.adj.ps	&	4.0833		&	4.3333		&	0.17322		&	0.6792	&	n.s.			\\ \hline

%sources		&	zscore F value	&	zscore p-value 	& 	raw F value	&	raw p-value	\\ \hline
%dem.non.ps	&	24.671		&	9.819e-06		&	20.323		&	4.484e-05		\\ \hline
%dem.adj.ps	&	16.844 		&	0.0001642	&	10.204		&	0.002531		\\ \hline
%num.adj.ps	&	25.024 		&	8.718e-06		&	15.713		&	0.000255		\\ \hline
%def.adj.ps	&	14.808		&	0.0003652	&	10.819		&	0.001931		\\ \hline
%pos.non.ps	&	140.77		&	1.332e-15		&	93.525		&	1.432e-09		\\ \hline
%pos.adj.ps	&	0.35563		&	0.5539		&	0.019377		&	0.8899		\\ \hline
%hisher.non.ps	&	36.517 		&	2.497e-07		&	32.522		&	8.057e-07		\\ \hline
%hisher.adj.ps	&	0.024874		&	0.8754		&	0.17322		&	0.6792		\\ \hline
%\end{tabular}}
%\caption{Plural Singular Results for Experiment 6 with no exclusion} \label{tab:10}
%\end{table} 

\subsubsection{Discussion}

Similar to Experiment 5, in 1S2P mismatch type, all conditions show a significant preference for the plural pivots. However, note that the plural pivot of dem.non.sp has a mean rating of 2.8 out of 7. With the mean rating of 1.3 for the singular pivot, this indicates that neither number marking is unacceptable for dem.non. The bare demonstratives resist CCA in mismatches is noted in Chapter 2. 

In 1P2S mismatch type, dem.non shows a significant preference for the singular pivot. However, similar to that in 1S2P, the mean ratings for dem.non are low: 3.8/7 for singular pivots and 1.5 for plural pivots, which indicates that the CCA is not available for dem.non

The singular pivot is acceptable in under dem.adj and num.adj, showing a CCA pattern. For pos.non, the plural pivot is accepted and the singular pivot is not. This is also predicted by the analysis in Chapter 2. Pos.adj and hisher.adj do not show significant preferences and the mean rating for both singular and plural pivots are around 4, i.e. borderline grammatical. This is not predicted by the analysis, however, this result converge with that in Experiment 5, which support the possibility that the availability of CCA in NRNR is sensitive to morphological marking on the sources. 

Def.adj shows a significant preference for plural pivots. Apart from the absence of morphological number marking which drives the rate for the singular pivot down, the possible coercion of \Next[a] to \Next[b] potentially drives the rating for the plural pivot up.

\ex. \a. The tall and the short student came from the U.S.
\b. The tall and short students came from the U.S.

Like Experiment 5, the hisher.non show a clear preference for plural pivots similar to pos.non. This again indicates that for at least some speakers, possessive pronouns are the same as the possessor DPs in that they can be conjoined. 

\subsection{Summary of Experiments on NRNR with Mismatching Values}

The two experiments on NRNR with mismatching values show converging results. Some predictions of the analysis from Chapter 2 have been confirmed: pivots under pos.non show plural agreement in both types of mismatches; pivots under dem.adj and number.adj show CCA. 

A possible new restriction on CCA is observed: CCA is triggered for sources that shows morphological marking, CCA under sources that does not show morphological agreement is rated lower e.g. pos.adj, hisher.adj, def.adj. Experiment 5 and 6 also show that some speakers treat hisher.non on the par of pos.non. 

\section{Conclusion}

In this chapter I reported 6 experimental studies on the nominal right node raising construction. 

The first three experiments use the forced choice task and the 7 point Likert scale task to test NRNR with two singular features. The predictions of the analysis proposed in Chapter 2 have largely been confirmed: except for bare possessors, other sources show a significant preference for singular pivots. The definite article presents an outlier in that it shows no significant preference. I propose that it could result from possible coercion by some speakers which deletes the definite article in the second source (possibly by inattenion). If this is the case, the definite article + adjective source is predicted to show a high rating for plural pivot in NRNR cases with both matching and mismatching values. This prediction is confirmed in Experiment 5 and 6. To further test the predictions the analysis in Chapter 3 for definite articles, one has to get rid of this possible coercion. I leave this to further research.

Experiment 4 uses 7 point Likert scale to test NRNR under different combinations of possessive pronouns. \cite{Shen:ip2} claim that NRNR is only licensed when the possessive pronoun in the first conjunct can license both overt and covert possessee, i.e. when the possessive pronoun in the first conjunct is \textit{his}. The experimental results are more complicated. No systematic differences have been observed between hisher.non and herhis.non. However, conditions herhis.non and herhis.adj, which are tested in both Experiment 3 and 4, end up with different results. Given that the materials and designs are identical between these two experiments, it remains unclear how this difference came up.

Experiment 5 and 6 use the forced choice task and the 7 point Likert scale task to test NRNR with one singular and one plural feature. The predicted CCA patterns under mismatch are found in sources with morphological agreement. Sources with no morphological agreement e.g. pos.adj, def.adj, hisher.adj do not show obvious CCA patterns. This leads to a possible restriction on CCA: CCA is triggered by morphological agreement. Syntactic agreement with no morphological reflex has a hard time triggering CCA.

One final finding involves experimental methodology. The contrast between Experiment 2 and 3 as well as Experiment 5 and 6 further support the claim made by \cite{Sprouse:2013} that the forced choice task is more sensitive to subtle effects than the 7 point Likert scale.

\bibliographystyle{spr-chicago}
\bibliography{Bibliography}

\end{document}


%%%%%%%%%%%%%%%%%%%%%%%%%%%%%%%%%%%%%%%%%%
%%%%%%%%%%%%%%%%%%%%%%%%%%%%%%%%%%%%%%%%%%

\noindent \textbf{2. Experiment 6: excluding people who had one or two control items wrong}

16 participants had one or both of the control items wrong and were excluded from this analysis. The remaining 368 participants are included in the analysis.

\begin{figure}[h!] 
\includegraphics[scale=0.15]{plot_7p3_head_violin_judgment_exl.png} \centering
\includegraphics[scale=0.15]{plot_7p3_tail_violin_judgment_exl.png} \centering
\caption{Plot for Experiment 6 with Exclusion}
\label{fig:ch3exp6c.violin}
\end{figure}


\begin{table}[htb!] \small \centering
\resizebox{0.8\textwidth}{!}
{
\begin{tabular}{ | c | c | c | c | c | c |} \hline 
sources		&	singular mean	&	plural mean 	& 	F value		&	p-value		&	Significance	\\ \hline
dem.non.sp	&	1.2			&	2.7			&	12.053		&	0.002166		&	**			\\ \hline
dem.adj.sp	&	1.7			&	4.8			&	35.375		&	4.028e-07		&	***			\\ \hline
num.adj.sp	&	1.7			&	6.4			&	186.14		&	2.2e-16		&	***			\\ \hline
def.adj.sp		&	3.6			&	5.8			&	15.005		&	0.0003377	&	***			\\ \hline
pos.non.sp	&	2.8			&	5.6			&	37.089		&	3.531e-07		&	***			\\ \hline
pos.adj.sp		&	3			&	5.8			&	32.522		&	8.057e-07		&	***			\\ \hline
hisher.non.sp	&	2.1			&	5.3			&	66.256		&	6.231e-08		&	***			\\ \hline
hisher.adj.sp	&	2.9			&	5.5			&	23.368		&	2.006e-05		&	***			\\ \hline
~			&	~			&	~			&	~			&	~			&	~			\\ \hline
dem.non.ps	&	3.8			&	1.5			&	20.323		&	4.484e-05		&	***			\\ \hline
dem.adj.ps	&	4.4			&	2.7			&	11.723		&	0.001347		&	**			\\ \hline
num.adj.ps	&	5.2			&	3			&	14.643		&	0.0004067	&	***			\\ \hline
def.adj.ps		&	3.8			&	5.6			&	11.145		&	0.001722		&	**			\\ \hline
pos.non.ps	&	1.9			&	5.8			&	93.525		&	1.432e-09		&	***			\\ \hline
pos.adj.ps		&	4.3			&	4.3			&	0.019377		&	0.8899		&	n.s			\\ \hline
hisher.non.ps	&	2.1			&	5.5			&	51.2			&	8.72e-09		&	***			\\ \hline
hisher.adj.ps	&	4			&	4.3			&	0.17894		&	0.6743		&	n.s			\\ \hline


%sources		&	zscore F value	&	zscore p-value 	& 	raw F value	&	raw p-value	\\ \hline
%dem.non.sp	&	17.308		&	0.0001451	&	12.053		&	0.002166		\\ \hline
%dem.adj.sp	&	49.329 		&	1.056e-08		&	35.375		&	4.028e-07		\\ \hline
%num.adj.sp	&	 260.76 		&	2.2e-16		&	186.14		&	2.2e-16		\\ \hline
%def.adj.sp		&	26.309		&	5.69e-06		&	15.005		&	0.0003377	\\ \hline
%pos.non.sp	&	56.764		&	3.419e-09		&	37.089		&	3.531e-07		\\ \hline
%pos.adj.sp		&	 55.335		&	2.01e-09		&	32.522		&	8.057e-07		\\ \hline
%hisher.non.sp	&	94.293 		&	3.233e-09		&	66.256		&	6.231e-08		\\ \hline
%hisher.adj.sp	&	28.07		&	4.559e-06		&	23.368		&	2.006e-05		\\ \hline
\end{tabular}}
\caption{Results for Experiment 6 with Exclusion} \label{tab:ch3exp6b}
\end{table} 

%\begin{table}[htb!] \small \centering
%\resizebox{0.8\textwidth}{!}
%{
%\begin{tabular}{ | c | c | c | c | c | c |} \hline 
%sources		&	singular mean	&	plural mean 	& 	F value		&	p-value		&	Significance	\\ \hline
%dem.non.ps	&	3.75			&	1.5417		&	20.323		&	4.484e-05		&	***			\\ \hline
%dem.adj.ps	&	4.3913		&	2.2609		&	11.723		&	0.001347		&	**			\\ \hline
%num.adj.ps	&	5.1739		&	3			&	14.643		&	0.0004067	&	***			\\ \hline
%def.adj.ps		&	3.7826		&	5.6087		&	11.145		&	0.001722		&	**			\\ \hline
%pos.non.ps	&	1.875		&	5.75			&	93.525		&	1.432e-09		&	***			\\ \hline
%pos.adj.ps		&	4.25			&	4.3333		&	0.019377		&	0.8899		&	n.s			\\ \hline
%hisher.non.ps	&	2.1364		&	5.5			&	51.2			&	8.72e-09		&	***			\\ \hline
%hisher.adj.ps	&	4			&	4.2609		&	0.17894		&	0.6743		&	n.s			\\ \hline

%sources		&	zscore F value	&	zscore p-value 	& 	raw F value	&	raw p-value	\\ \hline
%dem.non.ps	&	24.671		&	9.819e-06		&	20.323		&	4.484e-05		\\ \hline
%dem.adj.ps	&	19.889 		&	5.609e-05		&	11.723		&	0.001347		\\ \hline
%num.adj.ps	&	22.006 		&	2.656e-05		&	14.643		&	0.0004067	\\ \hline
%def.adj.ps		&	14.99		&	0.0003545	&	11.145		&	0.001722		\\ \hline
%pos.non.ps	&	140.77		&	1.332e-15		&	93.525		&	1.432e-09		\\ \hline
%pos.adj.ps		&	0.35563		&	0.5539		&	0.019377		&	0.8899		\\ \hline
%hisher.non.ps	&	62.685 		&	7.301e-10		&	51.2			&	8.72e-09		\\ \hline
%hisher.adj.ps	&	0.025639		&	0.8735		&	0.17894		&	0.6743		\\ \hline
%\end{tabular}}
%\caption{Plural Singular Results for Experiment 6 with exclusion} \label{tab:12}
%\end{table} 